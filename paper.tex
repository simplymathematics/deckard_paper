% Options for packages loaded elsewhere
\PassOptionsToPackage{unicode}{hyperref}
\PassOptionsToPackage{hyphens}{url}
%
\documentclass[
]{article}
\usepackage{amsmath,amssymb}
\usepackage{iftex}
\ifPDFTeX
  \usepackage[T1]{fontenc}
  \usepackage[utf8]{inputenc}
  \usepackage{textcomp} % provide euro and other symbols
\else % if luatex or xetex
  \usepackage{unicode-math} % this also loads fontspec
  \defaultfontfeatures{Scale=MatchLowercase}
  \defaultfontfeatures[\rmfamily]{Ligatures=TeX,Scale=1}
\fi
\usepackage{lmodern}
\ifPDFTeX\else
  % xetex/luatex font selection
\fi
% Use upquote if available, for straight quotes in verbatim environments
\IfFileExists{upquote.sty}{\usepackage{upquote}}{}
\IfFileExists{microtype.sty}{% use microtype if available
  \usepackage[]{microtype}
  \UseMicrotypeSet[protrusion]{basicmath} % disable protrusion for tt fonts
}{}
\makeatletter
\@ifundefined{KOMAClassName}{% if non-KOMA class
  \IfFileExists{parskip.sty}{%
    \usepackage{parskip}
  }{% else
    \setlength{\parindent}{0pt}
    \setlength{\parskip}{6pt plus 2pt minus 1pt}}
}{% if KOMA class
  \KOMAoptions{parskip=half}}
\makeatother
\usepackage{xcolor}
\usepackage[margin=1in]{geometry}
\usepackage{graphicx}
\makeatletter
\def\maxwidth{\ifdim\Gin@nat@width>\linewidth\linewidth\else\Gin@nat@width\fi}
\def\maxheight{\ifdim\Gin@nat@height>\textheight\textheight\else\Gin@nat@height\fi}
\makeatother
% Scale images if necessary, so that they will not overflow the page
% margins by default, and it is still possible to overwrite the defaults
% using explicit options in \includegraphics[width, height, ...]{}
\setkeys{Gin}{width=\maxwidth,height=\maxheight,keepaspectratio}
% Set default figure placement to htbp
\makeatletter
\def\fps@figure{htbp}
\makeatother
\setlength{\emergencystretch}{3em} % prevent overfull lines
\providecommand{\tightlist}{%
  \setlength{\itemsep}{0pt}\setlength{\parskip}{0pt}}
\setcounter{secnumdepth}{-\maxdimen} % remove section numbering
\newlength{\cslhangindent}
\setlength{\cslhangindent}{1.5em}
\newlength{\csllabelwidth}
\setlength{\csllabelwidth}{3em}
\newlength{\cslentryspacingunit} % times entry-spacing
\setlength{\cslentryspacingunit}{\parskip}
\newenvironment{CSLReferences}[2] % #1 hanging-ident, #2 entry spacing
 {% don't indent paragraphs
  \setlength{\parindent}{0pt}
  % turn on hanging indent if param 1 is 1
  \ifodd #1
  \let\oldpar\par
  \def\par{\hangindent=\cslhangindent\oldpar}
  \fi
  % set entry spacing
  \setlength{\parskip}{#2\cslentryspacingunit}
 }%
 {}
\usepackage{calc}
\newcommand{\CSLBlock}[1]{#1\hfill\break}
\newcommand{\CSLLeftMargin}[1]{\parbox[t]{\csllabelwidth}{#1}}
\newcommand{\CSLRightInline}[1]{\parbox[t]{\linewidth - \csllabelwidth}{#1}\break}
\newcommand{\CSLIndent}[1]{\hspace{\cslhangindent}#1}
\ifLuaTeX
  \usepackage{selnolig}  % disable illegal ligatures
\fi
\IfFileExists{bookmark.sty}{\usepackage{bookmark}}{\usepackage{hyperref}}
\IfFileExists{xurl.sty}{\usepackage{xurl}}{} % add URL line breaks if available
\urlstyle{same}
\hypersetup{
  pdftitle={Deckard: A Declarative Tool for Machine Learning Robustness Evaluations},
  hidelinks,
  pdfcreator={LaTeX via pandoc}}

\title{Deckard: A Declarative Tool for Machine Learning Robustness
Evaluations}
\author{}
\date{\vspace{-2.5em}15 October 2024}

\begin{document}
\maketitle

\hypertarget{summary}{%
\section{Summary}\label{summary}}

The software package presented, called \texttt{deckard}, is a modular
software toolkit designed to streamline and standardize experimentation
in machine learning (ML) with a particular focuse on the adversarial
scenario. It provides a flexible, extensible framework for defining,
executing, and analyzing end-to-end ML pipelines in the context of a
malicious actor. As it is built on top of the Hydra configuration
system, deckard supports declarative YAML-based configuration of data
preprocessing, model training, and adversarial attack pipelines,
enabling reproducible, framework-agnostic experimentation across diverse
ML settings.

In addition to configuration management, \texttt{deckard} includes a
suite of utilities for distributed and parallel execution, automated
hyperparameter optimisation, visualisation, and result aggregation. The
tooling abstracts away much of the engineering overhead typically
involved in adversarial ML research, allowing researchers to focus on
algorithmic insights rather than implementation details. The presented
software facilitates rigorous benchmarking by maintaining an auditable
trace of configurations, random seeds, and intermediate outputs
throughout the experimental lifecycle.

The system is compatible with a variety of ML frameworks and several
classes of adversarial attacks, making it a suitable backend for both
large-scale automated testing and fine-grained empirical analysis. By
providing a unified interface for experimental control, \texttt{deckard}
accelerates the development and evaluation of robust models, and helps
close the gap between research prototypes and verifiable, reproducible
results.

\hypertarget{statement-of-need}{%
\section{Statement of need}\label{statement-of-need}}

While tools such as \texttt{mlflow} (Zaharia et al. 2018), Weights \&
Biases (Biewald 2020), \texttt{optuna} (Akiba et al. 2019), and
Kubernetes (Kubernetes 2019) provide essential infrastructure for model
tracking and experiment management, \texttt{deckard} occupies a
different position in the ML ecosystem---focusing specifically on
configurable, adversarially robust experimentation.

Unlike MLflow and Weights \& Biases, which emphasize logging,
visualization, and reproducibility for various ML frameworks, deckard
enforces reproducibility by construction through its declarative,
YAML-driven configuration system built on Facebook's hydra (Yadan 2019)
configuration management tool. In contrast to cloud-management software
like Kubernetes---which is a general-purpose container orchestration
platform---\texttt{deckard} abstracts away orchestration details and
offers native support for parallel and distributed experimentation,
tailored to ML workflows involving attack/defense cycles, model
retraining, or optimisation. While deckard integrates tightly with IBM's
Adversarial Robustness Toolbox (Nicolae et al. 2018), the software is
designed to be easily extensible to other attack frameworks. The human-
and machine-readable parameter configuration system allows researchers
to declaratively define end-to-end pipelines that span data sampling,
preprocessing, model training, attack generation, defense evaluation,
multi-objective optimisation, and visualisation. Tools like \texttt{ray}
(Moritz et al. 2018), \texttt{optuna} (Akiba et al. 2019), or
\texttt{nevergrad} (Bennet et al. 2021) offer components of this
pipeline (\emph{e.g.}, hyperparameter search or configuration
management), but lack unified support for adversarial ML, verification,
or auditability at scale. While \texttt{deckard} complements these
existing tools, and in many cases can be integrated with them, its
primary contribution is in automating and verifying adversarial ML
experiments in a way that is both extensible and framework-agnostic.

\hypertarget{usage}{%
\section{Usage}\label{usage}}

Various versions of this software have been used in several recently
published and not-yet-published works by the author of this paper, all
of which are available in the \texttt{examples} folder in the source
code repository \url{https://github.com/simplymathematics/deckard}. One
published work, now reproducible via the
\texttt{examples/attack\_defence\_survey} folder, includes a large
survey of attacks and defences against canonical datasets and models (C.
Meyers, Löfstedt, and Elmroth 2023). Another work analysed the run-time
requirements of attacks against a particular model before and after
retraining against those attacks (C. Meyers, Löfstedt, and Elmroth 2024)
(reproducible via \texttt{examples/retraining}). The next paper
formalised a method for estimating the time-to-failure of a given model
against a suite of attacks and introduce a metric that quantifies the
ratio of attack and training cost (Meyers et al. 2023) (reproducible via
\texttt{examples/survival\_heuristic}). Furthemore, a not yet published
work uses this time-to-failure model as a mechanism for analysing the
cost efficacy of various hardware choices in the context of adversarial
attacks (reproducible via \texttt{examples/power}) (C. Meyers et al.
2024). Another work exploits the tooling to train a custom model that is
designed to run client-side by using compression algorithms to measure
the distance between text (reproducible via
\texttt{examples/compression}).

\hypertarget{experiment-management}{%
\section{Experiment Management}\label{experiment-management}}

Typically ML projects are composed of long and complex pipelines that
are highly dependent on a number of parameters that must be configured
by either the model builder or attacker. Due to the large scale and cost
associated with training ML models, it is often necessary to tune a
model using many indivudal model configurations (often called
\emph{hyper-paremeters}). To determine adversarial robustness, one of
many benchmark datasets is first sampled, then preprocessed, sent to a
model, with optional pre- and post-processing defences, and then scored
according to some chosen metric which may include the performance
against any number of adversarial attacks. Each stage in this example
pipeline might include tens or hundreds of possible sets of
hyper-parameters that must be exhaustively tested. Furthermore, this
problem scales drastically as we include more and more stages in a
pipeline since each additional stage introduces a new combinatorial
layer of complexity, rapidly expanding the total number of potential
configurations that must be evaluated for robustness and be reproducible
for posterity. Not only does \texttt{deckard} provide a standard way to
document and configure these hyper-parameters, it gives each experiment
an auditable identifier.

\hypertarget{reproducibility-and-auditability}{%
\section{Reproducibility and
Auditability}\label{reproducibility-and-auditability}}

For ML, various regulatory and legal frameworks govern safety (The
Parliament of the European Union 2024; {``{ISO} 26262-1:2011, Road
Vehicles --- Functional Safety''} 2018; \emph{IEC 61508 Safety and
Functional Safety} 2010; \emph{IEC 62304 Medical Device Software -
Software Life Cycle Processes} 2006), privacy (The Parliament of the
European Union 2024; European Parliament and Council of the European
Union 2016; Legislature of the United States 1996, 1998) and/or
transparency (The Parliament of the European Union 2024; The Legislature
of California 2024). The software package presented here provides a
machine- and human-readable format for creating reproducible and
auditable experiments as required by various regulations. In addition,
several examples connected to both published and not-yet-published work
live in the \texttt{examples} folder in the repository, allowing for
easy reproducibility of several extensive sets of experiments across
several popular ML software frameworks. The \texttt{power} example
provides a reproducible way to run a suite of adversarial tests using
popular cloud-based platforms and the \texttt{retraining} and
\texttt{survival\_heuristic} examples provide examples of both CPU and
GPU-based parallelisation, respectively.

The \texttt{basics} subfolder provides a minimum working example for
each of the supported ML frameworks: \texttt{tensorflow} (Abadi et al.
2015), \texttt{pytorch} (Paszke et al. 1912), \texttt{scikit-learn}
(Pedregosa et al. 2011), and \texttt{keras} (Chollet 2015). The basics
folder also provides examples of various classes of adversarial
examples: \emph{poisoning} attacks that change model behaviour by
injecting data during training (Biggio, Nelson, and Laskov 2012),
\emph{inference} attacks (Li and Zhang 2021) that attempt to reverse
engineer properties of the training data, \emph{extraction} attacks that
attempt to reverse engineer the model (Jagielski et al. 2020), and
\emph{evasion} attacks that induce errors of classification during
run-time (C. Meyers, Löfstedt, and Elmroth 2023). The parameters file
for each experiment ensures that a given pipeline can be reproduced and
the standardised format allows us to derive a hash value that is hard to
forge but easy to verify. Not only does this hash serve as an identifier
to track the state of an experiment, but also serves as a way to audit
the parameters file for tampering. Likewise, by using \texttt{dvc} (DVC
Authors 2023) to track any input or output files specified in the
parameters file, the software associates each score file with a
identifier that is easy to track and verify, but hard to
forge---ensuring that forged or modified results are easy to spot in
version-controlled experiment repository.

\hypertarget{parallel-and-distributed-design}{%
\section{Parallel and Distributed
Design}\label{parallel-and-distributed-design}}

Since ML projects can exploit specialized hardware such as multi-core
processors or GPUs, and often rely on clusters of machines for
large-scale data processing, it was necessary to enable parallel and
distributed experiment execution and model optimization. By leveraging
the \texttt{hydra} configuration framework, \texttt{deckard}
automatically supports optimization libraries like \texttt{nevergrad}
(Bennet et al. 2021), \texttt{Adaptive\ Experimentation} (A. Developers
2025), and \texttt{optuna} (Akiba et al. 2019), making the software
modular and extensible. Additionally, experiments can be managed using a
variety of popular job schedulers, including \texttt{Ray} (Moritz et al.
2018), \texttt{Redis\ Queue} (Stamps 2025), and \texttt{slurm} (Yoo,
Jette, and Grondona 2003) for distributd jobs or \texttt{joblib} (J.
Developers 2025) for jobs on a single machine.

By using a declarative design, a given set of experiments can be
specified once and executed seamlessly across different backends without
modification to the underlying codebase. This makes \texttt{deckard}
both adaptable and scalable, suitable for use on personal laptops,
multi-node servers, or large-scale, high-performance clusters. When
configured appropriately, experiment batches can be parallelized,
enabling massive parameter sweeps, ensemble evaluations, or adversarial
robustness tests to be executed in parallel---reducing turnaround time
while maintaining strong guarantees on reproducibility and auditability.
The design of the presented software prioritizes clarity and
maintainability by capturing each experimental configuration as a YAML
artifact, making both successful and failed runs equally traceable and
shareable. This approach transforms experiment tracking from an
afterthought into a first-class component of the trustworthy ML
workflow.

\hypertarget{funding}{%
\section{Funding}\label{funding}}

Financial support has been provided in part by the Knut and Alice
Wallenberg Foundation grant number 2019.0352 and by the eSSENCE
Programme under the Swedish Government's Strategic Research Initiative.

\hypertarget{acknowledgements}{%
\section{Acknowledgements}\label{acknowledgements}}

The author would like to thank Aaron MacSween, Abel Souza, and Mohammad
Saledghpour Reza for their guidance in software design principles. In
particular, the author appreciates Mohammad's code and documentation
regarding cloud-based and other Kubernetes deployments. The author would
like to thanks his advisors, Erik Elmroth and Tommy Löfstedt for their
research expertise, funding, and patience.

\hypertarget{references}{%
\section*{References}\label{references}}
\addcontentsline{toc}{section}{References}

\hypertarget{refs}{}
\begin{CSLReferences}{1}{0}
\leavevmode\vadjust pre{\hypertarget{ref-tensorflow}{}}%
Abadi, Martín, Ashish Agarwal, Paul Barham, Eugene Brevdo, Zhifeng Chen,
Craig Citro, Greg S. Corrado, et al. 2015. {``{TensorFlow}: Large-Scale
Machine Learning on Heterogeneous Systems.''}
\url{https://www.tensorflow.org/}.

\leavevmode\vadjust pre{\hypertarget{ref-optuna}{}}%
Akiba, Takuya, Shotaro Sano, Toshihiko Yanase, Takeru Ohta, and Masanori
Koyama. 2019. {``Optuna: A Next-Generation Hyperparameter Optimization
Framework.''} In \emph{Proceedings of the 25th ACM SIGKDD International
Conference on Knowledge Discovery \& Data Mining}, 2623--31.

\leavevmode\vadjust pre{\hypertarget{ref-nevergrad}{}}%
Bennet, Pauline, Carola Doerr, Antoine Moreau, Jeremy Rapin, Fabien
Teytaud, and Olivier Teytaud. 2021. {``Nevergrad: Black-Box Optimization
Platform.''} \emph{ACM SIGEVOlution} 14 (1): 8--15.

\leavevmode\vadjust pre{\hypertarget{ref-wandb}{}}%
Biewald, Lukas. 2020. {``Experiment Tracking with Weights and Biases.''}
\url{https://www.wandb.com/}.

\leavevmode\vadjust pre{\hypertarget{ref-biggio2012poisoning}{}}%
Biggio, Battista, Blaine Nelson, and Pavel Laskov. 2012. {``Poisoning
Attacks Against Support Vector Machines.''} \emph{International
Conference on Machine Learning}.

\leavevmode\vadjust pre{\hypertarget{ref-keras}{}}%
Chollet, François. 2015. {``Keras.''} \emph{GitHub Repository}.
\url{https://github.com/fchollet/keras}; GitHub.

\leavevmode\vadjust pre{\hypertarget{ref-ax}{}}%
Developers, Ax. 2025. {``AX: Adaptive Experimentation Platform.''}
\url{https://ax.readthedocs.io/en/stable/\#}.

\leavevmode\vadjust pre{\hypertarget{ref-joblib}{}}%
Developers, Joblib. 2025. {``Joblib: Running Python Functions as
Pipeline Jobs.''} \url{https://joblib.readthedocs.io/en/stable/}.

\leavevmode\vadjust pre{\hypertarget{ref-dvc}{}}%
DVC Authors. 2023. {``{DVC}--{Data Version Control}.''} Github.
\url{https://github.com/iterative/dvc.org}.

\leavevmode\vadjust pre{\hypertarget{ref-gdpr}{}}%
European Parliament, and Council of the European Union. 2016.
{``Regulation ({EU}) 2016/679 of the {European} {Parliament} and of the
{Council}. Of 27 {April} 2016 on the Protection of Natural Persons with
Regard to the Processing of Personal Data and on the Free Movement of
Such Data, and Repealing {Directive} 95/46/{EC} ({General} {Data}
{Protection} {Regulation}).''} OJ L 119, 4.5.2016, p. 1--88. May 4,
2016. \url{https://data.europa.eu/eli/reg/2016/679/oj}.

\leavevmode\vadjust pre{\hypertarget{ref-IEC61508}{}}%
\emph{IEC 61508 Safety and Functional Safety}. 2010. 2nd ed.
International Electrotechnical Commission.

\leavevmode\vadjust pre{\hypertarget{ref-IEC62034}{}}%
\emph{IEC 62304 Medical Device Software - Software Life Cycle
Processes}. 2006. 2nd ed. International Electrotechnical Commission.

\leavevmode\vadjust pre{\hypertarget{ref-iso26262}{}}%
{``{ISO} 26262-1:2011, Road Vehicles --- Functional Safety.''} 2018.
\url{https://www.iso.org/standard/43464.html} (visited 2022-04-20).

\leavevmode\vadjust pre{\hypertarget{ref-extraction_attack}{}}%
Jagielski, Matthew, Nicholas Carlini, David Berthelot, Alex Kurakin, and
Nicolas Papernot. 2020. {``High Accuracy and High Fidelity Extraction of
Neural Networks.''} In \emph{29th USENIX Security Symposium (USENIX
Security 20)}, 1345--62.

\leavevmode\vadjust pre{\hypertarget{ref-k8s}{}}%
Kubernetes. 2019. {``Kubernetes--an Open Source System for Managing
Containerized Applications.''} Github.
\url{https://github.com/kubernetes/kubernetes}.

\leavevmode\vadjust pre{\hypertarget{ref-hipaa}{}}%
Legislature of the United States. 1996. {``Health Insurance Portability
and Accountability Act.''}

\leavevmode\vadjust pre{\hypertarget{ref-coppa}{}}%
---------. 1998. {``Children's Online Privacy Protection Act.''}

\leavevmode\vadjust pre{\hypertarget{ref-li2021membership}{}}%
Li, Zheng, and Yang Zhang. 2021. {``Membership Leakage in Label-Only
Exposures.''} In \emph{Proceedings of the 2021 ACM SIGSAC Conference on
Computer and Communications Security}, 880--95.

\leavevmode\vadjust pre{\hypertarget{ref-meyers2023safety}{}}%
Meyers, Charles, Tommy Löfstedt, and Erik Elmroth. 2023.
{``Safety-Critical Computer Vision: An Empirical Survey of Adversarial
Evasion Attacks and Defenses on Computer Vision Systems.''}
\emph{Artificial Intelligence Review}, 1--35.

\leavevmode\vadjust pre{\hypertarget{ref-meyers2024massively}{}}%
---------. 2024. {``Massively Parallel Evasion Attacks and the Pitfalls
of Adversarial Retraining.''} \emph{EAI Endorsed Transactions on
Internet of Things} 10.

\leavevmode\vadjust pre{\hypertarget{ref-trashfire}{}}%
Meyers, Charles, Mohammad Reza Saleh Sedghpour, Tommy Löfstedt, and Erik
Elmroth. 2024. {``A Training Rate and Survival Heuristic for Inference
and Robustness Evaluation (TRASHFIRE).''}
\url{https://arxiv.org/abs/2401.13751}.

\leavevmode\vadjust pre{\hypertarget{ref-meyers_aft}{}}%
Meyers, Reza, Löfstedt, and Elmroth. 2023. {``A Systematic Approach to
Robustness Modelling.''} \emph{Springer Artificial Intelligence Review}.

\leavevmode\vadjust pre{\hypertarget{ref-ray}{}}%
Moritz, Philipp, Robert Nishihara, Stephanie Wang, Alexey Tumanov,
Richard Liaw, Eric Liang, Melih Elibol, et al. 2018. {``Ray: A
Distributed Framework for Emerging \(\{\)AI\(\}\) Applications.''} In
\emph{13th USENIX Symposium on Operating Systems Design and
Implementation (OSDI 18)}, 561--77.

\leavevmode\vadjust pre{\hypertarget{ref-art}{}}%
Nicolae, Maria-Irina, Mathieu Sinn, Minh Ngoc Tran, Beat Buesser,
Ambrish Rawat, Martin Wistuba, Valentina Zantedeschi, et al. 2018.
{``Adversarial Robustness Toolbox V1. 0.0.''} \emph{arXiv Preprint
arXiv:1807.01069}.

\leavevmode\vadjust pre{\hypertarget{ref-pytorch}{}}%
Paszke, Adam, Sam Gross, Francisco Massa, Adam Lerer, James Bradbury,
Gregory Chanan, Trevor Killeen, et al. 1912. {``Pytorch: An Imperative
Style, High-Performance Deep Learning Library. arXiv 2019.''}
\emph{arXiv Preprint arXiv:1912.01703} 10.

\leavevmode\vadjust pre{\hypertarget{ref-sklearn}{}}%
Pedregosa, F., G. Varoquaux, A. Gramfort, V. Michel, B. Thirion, O.
Grisel, M. Blondel, et al. 2011. {``Scikit-Learn: Machine Learning in
{P}ython.''} \emph{Journal of Machine Learning Research} 12: 2825--30.

\leavevmode\vadjust pre{\hypertarget{ref-rq}{}}%
Stamps. 2025. {``RQ: Easy Job Queues for Python.''} Stamps, an
Indonesian CRM company. \url{https://python-rq.org/}.

\leavevmode\vadjust pre{\hypertarget{ref-ai_pipeline_regulation}{}}%
The Legislature of California. 2024. {``AB-2013 Generative Artificial
Intelligence: Training Data Transparency.''}

\leavevmode\vadjust pre{\hypertarget{ref-ai_eu_act}{}}%
The Parliament of the European Union. 2024. {``High-Level Summary of the
AI Act.''}

\leavevmode\vadjust pre{\hypertarget{ref-hydra}{}}%
Yadan, Omry. 2019. {``Hydra -- a Framework for Elegantly Configuring
Complex Applications.''} Github.
\url{https://github.com/facebookresearch/hydra}.

\leavevmode\vadjust pre{\hypertarget{ref-slurm}{}}%
Yoo, Andy B, Morris A Jette, and Mark Grondona. 2003. {``Slurm: Simple
Linux Utility for Resource Management.''} In \emph{Workshop on Job
Scheduling Strategies for Parallel Processing}, 44--60. Springer.

\leavevmode\vadjust pre{\hypertarget{ref-mlflow}{}}%
Zaharia, Matei, Andrew Chen, Aaron Davidson, Ali Ghodsi, Sue Ann Hong,
Andy Konwinski, Siddharth Murching, et al. 2018. {``Accelerating the
Machine Learning Lifecycle with MLflow.''} \emph{IEEE Data Eng. Bull.}
41 (4): 39--45.

\end{CSLReferences}

\end{document}
